\documentclass{beamer}
\usepackage[utf8]{inputenc}
\usepackage{graphicx}
\usepackage{hyperref}

\usetheme{Berlin} % Zmiana motywu na Berlin
\usecolortheme{crane} % Zmiana koloru na crane

\title{Wspaniałe góry}
\author{Amelia Hajkowska}
\date{\today}

\begin{document}

\frame{\titlepage}

\begin{frame}
\frametitle{Spis treści}
\tableofcontents
\end{frame}

\section{Wprowadzenie}
\begin{frame}
\frametitle{Wprowadzenie}
Krótka historia o miłości do gór i ich odkrywaniu.
\end{frame}

\section{Rodzaje gór}
\subsection{Góry zrębowe}
\begin{frame}
\frametitle{Góry zrębowe}
Charakterystyka i przykłady gór zrębowych.
\begin{figure}
\includegraphics[width=0.5\textwidth]{gory1.jpg}
\caption{Góry zrębowe (chociaz tak naprawde nie}
\end{figure}
\end{frame}

\subsection{Góry wulkaniczne}
\begin{frame}
\frametitle{Góry wulkaniczne}
Opis i charakterystyka gór wulkanicznych.
\begin{itemize}
\item Rodzaje wulkanów
\item Przykłady gór wulkanicznych
\end{itemize}
\end{frame}

\section{Klimat górski}
\subsection{Warunki klimatyczne}
\begin{frame}
\frametitle{Warunki klimatyczne}
Jak klimat wpływa na życie w górach.
\end{frame}

\subsection{Flora i fauna}
\begin{frame}
\frametitle{Flora i fauna}
Bogactwo przyrodnicze gór.
\begin{table}
\centering
\begin{tabular}{|c|c|c|}
\hline
Rodzaj & Flora & Fauna \\
\hline
Góry zrębowe & Rodzaj A & A \\
Góry wulkaniczne & Rodzaj B & B \\
\hline
\end{tabular}
\caption{Flora i fauna w różnych typach gór}
\end{table}
\end{frame}

\section{Alpinizm}
\begin{frame}
\frametitle{Alpinizm}
Historia i rozwój alpinizmu.
\end{frame}

\section{Ochrona gór}
\begin{frame}
\frametitle{Ochrona gór}
Działania na rzecz ochrony środowiska górskiego.
\end{frame}

\section{Podsumowanie}
\begin{frame}
\frametitle{Podsumowanie}
Podsumowanie prezentacji i znaczenia gór w życiu człowieka.
\end{frame}

\section{Bibliografia}
\begin{frame}[allowframebreaks]
\frametitle{Bibliografia}
\begin{thebibliography}{9}
\bibitem{mountains} 
Jan Kowalski, 
\textit{Góry Świata}. 
Wydawnictwo Górskie, Warszawa, 2000.

\bibitem{alpinism} 
Anna Nowak, 
\textit{Historia alpinizmu}. 
Alpinizm Press, Kraków, 2010.
\end{thebibliography}
\end{frame}

\end{document}
