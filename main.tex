\documentclass{article}
\usepackage[utf8]{inputenc}
\usepackage{graphicx}
\usepackage{amsmath}
\usepackage{hyperref}

\title{stepik16_zad1}
\author{Amelia Hajkowska}
\date{29-01-2024}

\begin{document}

\maketitle

\begin{abstract}
Oto streszczenie pracy, które zawiera krótkie podsumowanie różnych tematów omówionych w dokumencie.
\end{abstract}

\tableofcontents
\listoffigures
\listoftables
\newpage

\section{Wstęp}
\subsection{Cel pracy}
Tekst wstępu do pracy. 
\subsection{Zobacz więcej}
Aby zobaczyć \textbf{więcej}, przejdź na następne strony.
\newpage

\section{Rozwój tematu}
\subsection{Podsekcja 1}
Tekst dotyczący pierwszego aspektu tematu. \textbf{Pogrubienie} tekstu.

\subsection{Podsekcja 2}
Tekst dotyczący drugiego aspektu tematu z zastosowaniem \underline{podkreślenia} i \textit{kursywy}.

\subsection{Podsekcja 3}
Przykład zastosowania trybu matematycznego w LaTeX:
\begin{equation}
    E = mc^2
\end{equation}
\begin{equation}
    a^2 + b^2 = c^2
\end{equation}

\begin{figure}[h]
\centering
\includegraphics[width=0.5\textwidth]{gory1.jpg}
\caption{Przykładowy rysunek 1}
\label{fig:rys1}
\end{figure}

\begin{figure}[h]
\centering
\includegraphics[width=0.5\textwidth]{gory2.jpg}
\caption{Przykładowy rysunek 2}
\label{fig:rys2}
\end{figure}

\begin{table}[h]
\centering
\begin{tabular}{|c|c|c|}
\hline
Kolumna 1 & Kolumna 2 & Kolumna 3 \\
\hline
Wartość 1 & Wartość 2 & Wartość 3 \\
Wartość A & Wartość B & Wartość C \\
\hline
\end{tabular}
\caption{Przykładowa tabela}
\label{tab:tabela1}
\end{table}
\newpage

\section{Podsumowanie}
W podsumowaniu odnoszę się do Rysunku \ref{fig:rys1} i Tabeli \ref{tab:tabela1}. I to tyle.

\begin{thebibliography}{9}
\bibitem{latexcompanion} 
Jan Kowalski, Anna Nowak. 
\textit{The \LaTeX\ Companion}. 
Warszawa, 1995.
 
\bibitem{einstein} 
Albert Einstein. 
\textit{Zur Elektrodynamik bewegter Körper}. (German) 
[\textit{On the Electrodynamics of Moving Bodies}]. 
Annalen der Physik, 322(10):891–921, 1905.
\end{thebibliography}

\end{document}
